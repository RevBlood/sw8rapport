\subsection{Implementation of Data Management}
\label{subsec:datamanagement}

\paragraph{Server}
As mentioned in \ref{subsubsec:backend}, \texttt{ServiceHelper} provides an easy to use interface for the application. To clarify the entire process of communicating with the backend, consider \ref{lst:getfavorites}. This example is all the code needed from the \texttt{Activity} perspective. The \texttt{ServiceHelper} is used to get "favorised recipes by account id" and returns a list of recipes when finished.

\begin{lstlisting}[language=java, label={lst:getfavorites}, caption={Sending a backend request from an Activity}]
Recipes = ServiceHelper.GetFavorisedRecipesByAccountId(AccountId);
\end{lstlisting}

The process of \texttt{GetFavorisedRecipesByAccountId} can be seen in \ref{lst:servicehelper}. Most of the work to be done is still abstract, and the method mostly provides the procedure on how to create the request properly - an account id is required, and the request should be sent to a specific address (Line 4). Furthermore it is a \texttt{Get} request.
In this case there is nothing that needs JSON serialization, as the backend only needs an account id to perform this action. The backend response is however put through the \texttt{JSONHelper} and parsed into a readable format. The \texttt{JSONHelper} methods are generic and will attempt to deserialize from/serialize to any class provided. For this example it uses the \texttt{response} string, encoded in JSON, and attempts to match it with the \texttt{Recipe} class. The serialization/deserialization is done with the help of an external library called jackson (made by fasterxml)\cite{jackson}.

\begin{lstlisting}[language=java, label={lst:servicehelper}, caption={GetFavorisedRecipesByAccountId from ServiceHelper}]
public static ArrayList<Recipe> GetFavorisedRecipesByAccountId(int accountId){
    String response = null;
    try {
        response = HTTPHelper.HTTPGet("http://" + ip + ":8000/RestService/Favorises/GetFavorisedRecipesByAccountId?accountId=" + accountId);
        System.out.println(response);
    } catch (Exception e) {
        e.printStackTrace();
    }
    ArrayList<Recipe> recipes = JSONHelper.DeserializeList(response, Recipe.class);
    return recipes;
}
\end{lstlisting}

The threading mentioned in \ref{subsec:datamanagement} is still not visible because it is done within the \texttt{HTTPHelper}. To continue the example from before, the method \texttt{HTTPGet} is called, which can be seen in \ref{lst:httpget}. Here, the actual threading happens. An instance of \texttt{HTTPGetTask} is created and executed, and whenever finished it is returned. \texttt{executeOnExecutor} ensures that the task will run, even if other \texttt{AsyncTask} instances are running. Not doing so could block the application from continuing, if the \texttt{ServiceHelper} is used from another \texttt{AsyncTask}. This is also known as a deadlock, since the system effectively prevents both threads from executing.

\begin{lstlisting}[language=java, label={lst:httpget}, caption={Starting a thread for HTTPGet}]
public static String HTTPGet(String url) {
	HTTPGetTask getTask = new HTTPGetTask();
   	getTask.executeOnExecutor(AsyncTask.THREAD_POOL_EXECUTOR, url);
    try {
    	return getTask.get();
    } catch (InterruptedException e) {
        e.printStackTrace();
    	return null;
    } catch (ExecutionException e) {
        e.printStackTrace();
        return null;
    }
}
\end{lstlisting}