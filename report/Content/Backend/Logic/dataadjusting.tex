\subsection{Data Adjustment}
\label{subsec:dataadjustment}

\paragraph{Adapting the data}
As mentioned in section \ref{subsec:backend_datamanagement}, the data on ingredient offers are extracted from eTilbudsAvis. eTilbudsAvis offers information on \textit{Dealers}, \textit{Stores} and \textit{Offers}, and we have chosen to gather all the data in individual files because of practical reasons, as it does take quite some time to gather all the data.  We have created models that matches the information models on eTilbudsAvis, and these can be seen in \texttt{eTilbudsObjects}.

Once we have the data in models matching their models, it might seem trivial to convert it to the data models we initially have for our database. Some calculations needed is calculating the savings both in price and percent, but that is simple mathematics. It does however have a lot of faulty data and null values, as the stores does not always list pre-sale prices. This problem is circumvented by the code example below.

\begin{lstlisting} [language=c++]
 //Some items does not have values; directly, or because they can't be calculated: give default value of -1.0
                    if (String.IsNullOrEmpty(offers[i].pricing.pre_price.ToString())) {
                        dbOffers.GetOrSetNormalPrice = defaultValue;
                        dbOffers.GetOrSetKrSaving = defaultValue;
                        dbOffers.GetOrSetPercentageSavingGeneral = defaultValue;
                        dbOffers.GetOrSetPercentageSavingRetailer = defaultValue;

                    //If variables have values, calculate and/or convert them to an appropriate format.
                    } else {
                        dbOffers.GetOrSetNormalPrice = (decimal)offers[i].pricing.pre_price;
                        kroner_savings = dbOffers.GetOrSetNormalPrice - dbOffers.GetOrSetOnSalePrice;
                        dbOffers.GetOrSetKrSaving = kroner_savings;
                        percent_saving_retailer = (dbOffers.GetOrSetOnSalePrice / dbOffers.GetOrSetNormalPrice) * 100;
                        dbOffers.GetOrSetPercentageSavingRetailer = percent_saving_retailer;

                        //TODO: once the database has been set up and each offer has been matched with ingredient, do this.
                        dbOffers.GetOrSetPercentageSavingGeneral = (decimal)-1.0;
                    }
\end{lstlisting}

A lot of the data is completely irrelevant to our database, this is quickly weaved away. The code for adapting along with gathering the data does take a while to run. This fact together with the data not changing during the daytime, is why the idea is to update the data at night.

As for the default data from Statistics Denmark, the data is supplied in a comma separated file. Converting this data to our models is a trivial task, however the data does have some shortcomings and has forced us to revitalize our ER-diagram to a certain extend. The code for extracting the data is located in the class-file \texttt{Script.cs} and the data can be seen in \ref{appendix:averageprices}.
