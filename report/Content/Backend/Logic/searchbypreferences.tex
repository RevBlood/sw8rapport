\subsection{Recommending By Preferences}
\label{subsec:searchbypref}

One of the biggest features in the project is the functionality of recommending recipes based on preferences. As per right now, the preferences you can adjust is how much focus you put on price and how much you focus you put on savings. These are represented as sliders in the personal tab on the application. The value of the sliders are relative to each other, meaning if you set the price slider at 10 and the savings on 5, price will count for 2/3' of the maximum preferences. This means that price gets a preference multiplier of 0.66, while savings gets 0.33.

The method has been split into different sections.

\paragraph{getExtraData} This entire method is used several places in the application. It has the purpose of getting the ExtraData for lists of recipes to display, such as in the favorites and discover fragments in the application. More importantly for the section, it is the first stepping stone in the recommending recipes by preferences.

The initial part of searching through recipes is as mentioned getting stores within a radius set by the user. We created a lambda expression built on GeoCoordinates.

\begin{lstlisting}[language=c++]
 public List<Retailer> GetStoresWithinRadius(List<Retailer> retailers, int radius, double latitude, double longitude) {

            GeoCoordinate center = new GeoCoordinate(latitude, longitude);

            return retailers.Select(x => new KeyValuePair<Retailer, GeoCoordinate>
				(x, new GeoCoordinate(Convert.ToDouble(x.GetOrSetLatitude), Convert.ToDouble(x.GetOrSetLongitude))))
                                 .Where(x => x.Value.GetDistanceTo(center) < radius).ToDictionary(x => x.Key, x => x.Value).Keys.ToList();
        }
\end{lstlisting}

After getting the stores, they are crossed with offers to find only the stores who offer anything related to the recipe. All of the ExtraData is calculated from looking into the ingredients of each recipe and adding costs and applying math. As the method dictates we return a dictionary of all the retailers for the specific recipe, and all of the ExtraData correlated to it.

\paragraph{sortBasedOnPreferences}

The idea is that a recipe is given a value between 1-100 based on preferences relative to the values of other recipes in the database. Lets say a recipe cheaper than the average is given a higher value, with the cheapest being given 100 points. These points are then multiplied using the preference multiplier, and saved as the rating in ExtraData.

The first part is getting all the recipes possible from within the area and the ExtraData correlated to it. We want to avoid having the same recipe from 5 different stores in area, so we make sure that only the best suggestion for each recipe is shown, based on the preferences. So the initial problem is getting the best suggestion based of the retailers, which is why we created this generic method to rate based on the ExtraData.

\begin{lstlisting} [language=c++]
        private Dictionary<T, ExtraData> RangeEvaluator<T>(Dictionary<T, ExtraData> list, int totalPartitions, int preference) {
            int recipeCount = list.Count;
            int tempInt = 0;
            for (int i = totalPartitions; i >= 0; i--) {

                // Calculate range of entries that should have a certain rating and remember how much was subtracted (for next iteration)
                int partitionSize = Convert.ToInt32(Math.Round((double)recipeCount / i));
                recipeCount -= partitionSize;

                // Fill in the relevant rating inside the calculated range of the list
                for (int k = tempInt; k < tempInt + partitionSize; k++) {
                    tempInt += partitionSize;
                    list.ElementAt(k).Value.Rating += i * preference;
                }

            }
            return list;
        }
\end{lstlisting}

Extracting the recipe for each recipeId, with the highest rating will ensure that only one of each recipe will be counted in the final recommendations. The next step is running the generic method over each recipe and sorting based on rating again. The final step is limiting the amount to match the setting given by the user.