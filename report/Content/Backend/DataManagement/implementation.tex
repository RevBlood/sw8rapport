\subsection{Implementation of Database}
\label{subsec:dbImpl}
In this project we implemented our database through an open source database called \textit{PostgreSQL}\cite{postgreSQL}, version 9.3. We previously worked with this database during a course called \textit{Database System} and in previous semesters as well. Following the database technology follows an administration tool called \textit{pgAdmin}\cite{pgAdmin}, which provides an easy to use GUI-based platform to work with everything from schemas and tables to SQL-queries and data-visibility. We use \textit{pgAmin} version 1.81.1.

It was already clear what tables and attributes we needed to implement, this can be seen in the data management design-subsection \ref{subsec:datamanagement}. Due to experience with the initial steps of database design and implementation, we knew that alterations without a doubt would occur. This usually results in remaking the database by changing schemas or alternating the structure of different tables. We wanted to make these alterations as inexpensive as possible. If we would have entered our database design through the administrative tools for \textit{PostgreSQL}, this would have become expensive and time-consuming, however we decided to use another solution. We wrote the SQL commands to create the entire database in a single text-document, allowing the use of a function in \textit{pgAdmin} which executes arbitrary SQL queries on the database. By using this approach, we are able to dump the entire database and restore it to a clean state in seconds. Additionally, we can alter the text-document if we want to make changes to the database and test it on another computer before using it on the functioning database on the backend. The complete text-document for creating the database can be seen in appendix \ref{appendix:createdatabase}. 

As stated in the introduction to the backend \ref{chap:backend}, the backend functionality is programmed in C\# . This means we want to query the database through the C\# code, and to do this we used a library called \textit{Npgsql}\cite{npgsql}. It is a .Net Data Provider for PostgreSQL, and it is developed to access the database server from the .Net framework\cite{npgsql}. 

The backend queries the database through a class called \texttt{DBController}. This class is approximately 700 lines of code, containing connecting establishment as well as select/insert/update/delete SQL-queries. \texttt{DBController} contains a total of 55 method, most of which changes the state of the database. An example showing how to add a new \textit{Account} to the Database can be seen in listing \ref{lst:DBAddAccount}. It will described below:

\begin{lstlisting}[language=c++, caption=Adding a new Account to the Database, label={lst:DBAddAccount}]
public int AddAccount(Account accountToInsert) {
	string sql = "INSERT INTO accounts(email, password, alias, settings, preferences) VALUES (@email, @password, @alias, @settings, @preferences)";

	NpgsqlCommand command = new NpgsqlCommand(sql, conn);
	command.Parameters.AddWithValue("@email", accountToInsert.GetOrSetEmail);
	command.Parameters.AddWithValue("@password", accountToInsert.GetOrSetPassword);
	command.Parameters.AddWithValue("@alias", accountToInsert.GetOrSetAlias);
	command.Parameters.AddWithValue("@settings", accountToInsert.GetOrSetSettings);
	command.Parameters.AddWithValue("@preferences", accountToInsert.GetOrSetPreferences);
	
	return NonQuery(command, "accounts");
}
\end{lstlisting}

In this piece of code, we build the SQL-statement that we want to execute on the database. We then make use of \textit{Npgsql} and uses their class called \texttt{NpgsqlCommand}, which takes parameter our SQL-statement, and the connection object initialized when we created an instance of \texttt{DBController}. We proceed to add the values we want to insert into the database, provided by the \texttt{Account} object sent as parameter to the method call. The last thing we do is return the result of a method call to \texttt{NonQuery}. \texttt{NonQuery} executes the desired command on the database, and it is targeted on the second parameter, in this case called \texttt{accounts}. What is returned in this case, is the amount of altered rows in the database, in this case one if the query was successful.

Another great feature about \textit{Npgsql} is their use of \texttt{DataRows}\cite{datarow}. Whenever we query the database with select-statements, it is returned as \texttt{DataRowCollection}, e.g. a collection of objects of type \texttt{DataRow}. This allows for easy creation of C\# objects, because we can make a specific constructor to a model following the same principles. An example is shown below in listing \ref{lst:DBGetAccount}, showing how an object of the class \texttt{Account} is instantiated from a \texttt{DataRow} object:

\begin{lstlisting}[language=c++, caption=Instantiating a new Account-object from a DataRow-object, label={lst:DBGetAccount}]
public Account(DataRow row) {
	this.GetOrSetId = row.Field<int>("accountid");
	this.GetOrSetEmail = row.Field<string>("email");
	this.GetOrSetPassword = row.Field<string>("password");
	this.GetOrSetAlias = row.Field<string>("alias");
	this.GetOrSetCreationDate = row.Field<DateTime>("creationdate");
	this.GetOrSetSettings = row.Field<string>("settings");
	this.GetOrSetPreferences = row.Field<string>("preferences");
}
\end{lstlisting}

This procedure is repeated for each select-statement made to the database for each of our model and relationships classes.

