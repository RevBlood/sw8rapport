\subsection{Design of the dataflow}\label{subsec:dataflow}

\textbf{work in progress}

Gathering and structuring the data is quite a nuisance in a project like ours. The stores and retailers do not put official prices of everything, they only issue a catalog once a month, with the offers  they have. \figref{fig:datavflow} shows the absolute overall design of the dataflow, and it illustrates the trouble of gathering information. The recipes have to be gathered from a free website or as our ER-diagram shows, users upload recipes. Opskrifter.dk is a huge online database with around 3.000 different recipes, and these are the recipes we intend to use and work with in our application. This part is rather trivial, however adjusting recipes for spell errors and matching them with the right ingredients can be problem, and requires some string analysis and management.

The Ingredient data is something entirely more complicated, and as mentioned retailers do not issue the full prices. We made an agreement with Statistics Denmark for the average prices of groceries across stores and retailers and planned to use these as a base for all calculations. The offers for the different groceries are gathered through eTilbudsAvisen. It is an online service which gatheres all offers offered by a selection of retailers in Denmark, and it does so by traversing the catalogs and extracting all the data. They offer all their data through a web Api, and the idea is for us to download their entire database once a day, and make the calculations necessary for it to match our database. 

\begin{figure}
\label{fig:dataflow}
\centering
\includegraphics[width=0.95\textwidth]{Pictures/dataflow}
\caption{A model of the dataflow through the entire system}
\end{figure}

\paragraph{Design of the Database}
\label{para:dbdesign}

The database is major part of the server, and a big part of the application is dependent upon the database being frequently and correctly updated. Furthermore the database has to be designed in a way that makes calculations fast to access and compute, as we do not want the user to wait for long queries and joins in order to get something shown on the application. 

\begin{figure}
\label{fig:ER-diagram}
\centering
\includegraphics[width=0.95\textwidth]{Pictures/ERdiagram}
\caption{The final ER diagram of the database}
\end{figure}

A lot of these relations speak for themselves, and the entire amount of relationships between account and recipe rather intuitive. We have chosen not to gather all the data in a single relationship as we want perform quick queries on the database rather than save memory. This is one of the advantages of having a stationary server with a large amount of memory available, it is possible to focus solely on performance. 

The biggest table in terms of sheer numbers would be ingredientIn, as each recipe has a lot of different ingredients. This has been accommodated for by moving most attributes to either recipe or ingredient, and we make sure to query through IngredientIn before joining any of these together.

