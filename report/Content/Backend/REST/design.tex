\subsection{Design of a RESTful Web Service}
\label{subsec:restDes}

\textbf{RESTful introduction}\newline

Representational State Transfer is an architectural style designed to offer services on the Web. A service is a software component provided through a network-accessible endpoint. The architectural style considers data and functionality as resources and allow access to them through Uniform Resource Identifiers (URIs). The data and functionality can be accessed by using a simple set of operations, typically four methods: PUT, GET, POST, and DELETE. If you want to use REST, you are constrained to use a client/server architecture and to use a stateless communication protocol, which typically will be HTTP. In this architecture, clients and servers exchange representations of resources by using a standardized interface and protocol. The following four principles are encouraged in RESTful applications, to keep them simple, lightweight and fast\cite{WhatAreRESTful}\cite{choosingRESTful}:
\begin{itemize}
\item Resource identification through URI
\item Uniform interface
\item Self-descriptive messages
\item Stateful interactions through hyperlinks
\end{itemize}

RESTful Web services are considered simple because they are built upon well-known standards(HTTP, XML, URI, MIME). REST gained increasing attention because of its usage in APIs in Web 2.0 services, but also because of the simplicity it offers in publishing and consuming a Web Service. Considering the possible complexity of designing a Web service interface, REST is simple because it completely limits the set of operations. The requirements to build a client to test a RESTful service is very small because it can be accessed through a regular browser. Deploying a RESTful Web service is compared to building a dynamic Website, which can be considered doable with limited resources, limitations(like programmers), time and skill. 

\textbf{Why did we choose to use a RESTful web service}
In the design of our application we decided to remove data-storage and larger computations from the mobile application, and push this back to a server. We did this to limit the resources needed by the mobile device. But the data and computations are worth very little if the mobile application cannot access it and retrieve it for the users. Therefore we needed an interface that could be used to link the client and server together, and this is why we chose to build a web service.

RESTful is based on HTTP requests, which make the server-interface uniform for multiple types of clients. We do not have to make separate interfaces for browsers, android, IOS, windows phone or others. This is very desirable for our backend-services and a large reason why we chose to design and implement a web service. 

Beside the uniform interface in RESTful, we also chose this specific type of web service because it is easy to make, and follows an ad-hoc type of style\cite{choosingRESTful}. 

We chose JSON for our message format. It is lightweight and is heavily supported by multiple build-in and external libraries. We also worked with this format before and felt it would fit this project. We also did not want to spend excessive time on designing our own plain text communication protocol.

\todo{lav de refs der skal til}

\todo{exposure of method+data design}