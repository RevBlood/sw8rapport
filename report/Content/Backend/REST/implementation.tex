\subsection{Implementation of RESTful Web Service}
\label{subsec:restImpl}

We decided to write our RESTful Web Service in C\# because this is a programming language we are very familiar with. C\# also offers a wide array of libraries to use, and in this case we used a collection called \texttt{System.ServiceModel.Web} which contains a specifically useful class for developing a web service, namely a class called \texttt{WebServiceHost}. The \texttt{WebServiceHost} is derived from the general C\# collection of \texttt{ServiceHost} and compliments the Windows Communication Foundation (WCF) REST programming model\cite{WebServiceHost}.
WCF is a framework made for building service-oriented applications. WCF offers asynchronous communications between endpoints, which in our example is a client requesting data from a service endpoint. WCF offers a long range of features, and the two most important that we use in this project are \textit{Data Contracts} and \textit{REST support}\cite{WCF}. 
With \textit{Data Contracts} you create a formal agreement between a service and a client that abstractly describes the data to be exchanged. A \textit{Data Contract} precisely defines what data is serialized and sent to the client. The serialization can be whichever format you desire, but XML and JSON are mainly used. As stated before, we chose to use JSON because we can use libraries to handle this, not only on the backend, but also in the application when the client has to deserialize the data.

There are three major steps you need to complete to implement a RESTful Web Service. The steps will be elaborated on below the listing:
\begin{itemize}
\item Define the URI to host on
\item Create an instance of the WebServiceHost class and a class which implements your desired service interface. This class needs to define a ServiceBehavior.
\item Add the desired Service endpoints, and attach an interface, the binding type and the URI on which you want to host the specific interface.
\end{itemize}

We define a URI with the C\# class called \texttt{URI}\cite{URIclass}, in which you define the path to your web service. We host our web service on the computers WLAN IPv4 address, and define the port to be 8000. There is no major reason for choosing this port number, we just wanted an unused port. In our project the URI looks like this: \texttt{http:// + [localIP] + :8000/RestService}. The RestService tag is just a descriptive subdomain, and it could alternatively be called API or likewise. One issues you have to address before you can host on this URI-path, is the case of running it on a Microsoft Windows machine - you need access rights to the URI namespace. If you do not have this in C\#, you will receive an AddressAccessDeniedException - therefore you need to register this URI. You do that by using the network shell, called netsh, using the following command: \texttt{netsh http add urlacl url=http://+:8000/ user=DOMAIN\textbackslash UserName"}. The command registers an HTTP address to the specified user, disregarding administrator rights\cite{netshURI}.

A WebServiceHost is instantiated through giving it two parameters, namely a class containing a service behavior definition, and the created URI on which we host the web service. In our project we named this \texttt{ServiceBehavior}\cite{ServiceBehavior} class \texttt{RestService}, and it implements all the interfaces we want to expose in our web service. \texttt{RestService} defines two properties, \texttt{InstanceContextMode}\cite{InstanceCtxMode} and \texttt{AddressFilterMode}\cite{AddressFilterMode}. \texttt{InstanceContextMode} is set to \textit{Single}, which means a new \texttt{InstanceContext} object is created on each incoming call, and this object is never reused. \texttt{AddressFilterMode} is set to \textit{Any}, which allows every address to access the web service. If you want to deny any specific address from accessing your services, you can allow only exact addresses through. 

The last thing you have to do before you are done setting up your web service is to add service endpoints. A service endpoint needs three parameters, an interface, a binding-type, and a URI. The interface needs to contain the method definitions of the desired functionality in the web service. The functionality inside these methods will be added in the \texttt{RestService}, which implements the interface. The binding-type in our project is \texttt{WebHttpBinding}\cite{WebHttpBinding}, because we want to expose our web service through HTTP requests. Finally, we define a URI on which the specific functionality in the specified interface can be accessed. In our project it looks like this: \texttt{new Uri(uri + "/Account")}, where the uri is the root-URI defined above.

Now, hosting a web service without functionality makes no sense. We have built a total of 11 different interfaces, which each have their own endpoint in our web service. Those 11 interfaces exposes a total of 46 different HTTP requests. 

Functionality is added to the web service by defining the interface as a \texttt{ServiceContract}\cite{ServiceContract}. Inside the \texttt{ServiceContract} each method is defined as an \texttt{OperationContract}\cite{OperationContract}, and it is these that the client may access. The \texttt{ServiceContract} in combination with the \texttt{WebHttpBinding} exposes the \texttt{OperationContract's} to the client through HTTP requests. An example can be seen at listing \ref{lst:servicecontract}.

\begin{lstlisting}[language=c++, label={lst:servicecontract}, caption=Definition of a ServiceContract]
[ServiceContract]
public interface IAccount {
   [OperationContract]
   void AddAccount(Account acc);
}
\end{lstlisting}

Additionally, since the interface now contains methods, the \texttt{RestService} class must implement the interfaces, if not it will result in thrown exceptions. This means that the \texttt{RestService} class must implement a method corresponding to the definition made in the interface, for all 46 methods, but this alone does not expose functionality to the user. You need to use the \texttt{WebInvoke}\cite{WebInvoke} class, still from the C\# \texttt{System.ServiceModel.Web}, and you need to define a HTTP method, URI template, a request format and a response format. We work primarily with three HTTP methods in our project, GET, POST, and DELETE. The URI template has to be a describing name corresponding to the functionality beneath. Finally, the message formats decided for this project is JSON. This is elaborated upon in the RESTful design, see section \ref{subsec:restDes}. An example showing \texttt{WebInvoke} can be seen at listing 

\begin{lstlisting}[language=c++, label={lst:webinvoke}, caption=Definition of a WebInvoke]
[WebInvoke(Method = "POST", UriTemplate = "AddRecipe", 
 RequestFormat = WebMessageFormat.Json, 
 ResponseFormat = WebMessageFormat.Json)]
\end{lstlisting}

\texttt{WebInvoke} ensures that a method can be invoked, and that it can be called and used through the WCF REST programming model. This completes the implementation of our RESTful web service. This procedure has been repeated continuously for each 46 HTTP request made in the web service.