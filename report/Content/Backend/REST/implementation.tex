\subsection{Implementation of RESTful Web Service}
\label{subsec:restImpl}

We decided to write our RESTful Web Service in C\# because this is a programming language we are very familiar with. C\# also offers a big library you can use, and in this case we used a collection called \texttt{System.ServiceModel.Web} which contains a specifically useful class for developing a web service, namely a class called \texttt{WebServiceHost}. The \texttt{WebServiceHost} is derived from the general C\# collection of \texttt{ServiceHost} and compliments the Windows Communication Foundation (WCF) REST programming model \todo{fix ref}.
%https://msdn.microsoft.com/en-us/library/system.servicemodel.web.webservicehost%28v=vs.110%29.aspx
WCF is a framework made for building service-oriented applications. WCF offers asynchronous communications between endpoints, which in our example is a client requesting data from a service endpoint. WCF offers a long range of features, the two most important that we use in this project is \textit{Data Contracts} and \textit{REST support}\todo{fix ref}. 
% https://msdn.microsoft.com/en-us/library/ms731082%28v=vs.110%29.aspx
With \textit{Data Contracts} you create a formal agreement between a service and a client that abstractly describes the data to be exchanged. A \textit{Data Contract} precisely defines what data is serialized and sent to the client. The serialization can be whichever format you desire, but XML and JSON are mainly used. As stated before, we chose to use JSON because we can use libraries to handle this, not only on the backend, but also in the application when the client has to deserialize the data.

There are three major steps you need to complete to implement a RESTful Web Service. The steps will be elaborated on below the listing:
\begin{itemize}
\item Define URI to host on
\item Create an instance of the WebServiceHost class and a class which implements your desired service interface. This class needs to define a ServiceBehavior.
\item Add the desired Service Endpoints, attaching an interface, binding type and the uri on which you want to host the specific interface.
\end{itemize}

We define a URI with the C\# class called \texttt{URI}, which 

- interface, 
- WebHttpBinding, exposing a web service through HTTP requests. each service much be set up with WebGetAttribute and WebInvokeAttribute, to get and post data.
- The url to host the specific interface on

Write here man