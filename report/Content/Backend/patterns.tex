\section{Patterns}
\label{sec:bpatterns}
Software patterns may provide consistency and structure in the code-project. We are following a small list of patterns during the development of the backend, as described below:

\paragraph{Data Access Object}
Naturally in the creation of the backend we have to work with data. The data management is further described in section \ref{sec:datamanage}. During the design and implementation, we want to follow the DAO pattern. The pattern describes that a single object should be used for abstracting away the data access from other layers. Following this pattern allows a programmer working on other parts of the backend to utilize the database functionality without being exposed unnecessary details. This ensures a single module handles the communication layer between the database and the rest of the backend.

\paragraph{Lazy Initialization}
It is natural for a programmer to try to preserve resources and at least consider not misusing them. Lazy initialization attempts to help this preservation by delaying the creation of objects and calculations as much as possible. The ideal case is that they will only be computed right before they are used, and in some cases avoids work that would have been canceled before use anyway.

\paragraph{Facade}
The facade pattern attempts to encapsulate a library or APIs behind a better designed facade. In the case of our backend, we attempt to make a uniform and well-structured facade that allows the client to work with data-storage and heavy computations, without knowing the inner workings of these. This enables a smaller training session to use complex technology, compared to a client who has to look behind the facade and know exactly how the data-storage and computations are functioning.
