\subsection{Design of the Models}\label{subsec:designmodels}
It is desirable to have functioning models in the C\# code that correspond to the models build in the database. This will allow us to work with the data from the database and perform the requested computations to display what the user wanted to see. Additionally, we can make use of our models to ensure that we have a uniform data-structure in the database, by checking that the data is following a given standard. This goal is achieved by receiving the data a user wants stored in the database and instantiate the given data as models on the backend. If the models are created successfully, it can be used to make an insertion into the database. If the creation of the models fail, we can provide the user with an error-message, that will help them correct their mistakes. This can be achieved by defining proper get- and setters for the properties in each model-class. 

An example of this could be disallowing the user to choose a username that starts with a number instead of a letter. Another example could be the requirement that a password is six characters long and contain at least one number.

By doing this uniform data-checking on the backend, you create an exposed interface to the programmers using the interface that will first: take care of this error-checking for them, and second: ensure a uniform platform independent data-structure. This data-structure will look exactly the same way and follow the same set of rules regardless of the platform and programming-language. 