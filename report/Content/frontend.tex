\chapter{Frontend}
\label{chap:frontend}

For the project to be useful, we needed the frontend to be accessible by people. The choice fell on Google's Android as the platform. There are several reasons that this choice was ideal for the project. The fact that it is designed for mobile is a must have. For an application like this there is no reason to assume that a person would always have a PC or similar device nearby when wanting to use the application. A smartphone running Android is however likely to be right in the pocket for many potential users. Other reasons to choose Android as platform are the groups previous experience, and both our mobile and developer equipment which consists of Android phones and Windows PC's. This enabled the use of Google's IDE Android Studio and live application testing on our own devices.

For the application itself we have weighted some qualities over others. Essentially we wanted to maximize the size of the application to make it as feature-strong and user-friendly as possible. This choice is simply based on the fact that this was a semester project, and that the application should be possible to demonstrate, rather than for example tested thoroughly for security issues or possible performance improvement.

In the end the application was only finished to a point, even for the qualities focused on. As such, this chapter will describe the application as-is at the end of the semester project. There are room for both new functionality and more testing and improving on current ones. 