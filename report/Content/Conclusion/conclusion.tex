\section{conclusion}
\label{sec:conc}

The final product is somewhat functional, it does give a suggestion based on the users preferences and allows for the user to easily gain information about nearby stores and retailers. The problemstatement is described in \ref{sec:probstate} and in the clearest meaning it has been solved, the application gives a suggestion based on the preferences set by the specific user. However the only preference to set at this given moment is monetary, we would have loved to base some preferences with more diversity so we could broaden our target group. We have as mentioned made it possible for each user to determine what the ideal radius for shopping is for them -this is perfect for fitting a broader group of people, it does not matter whether you have car or are walking.

The server has been developed with performance in mind. We have intentionally created some suboptimal spacial entities with the purpose of creating faster queries. The server does however spend some time making the calculations depending on how many recipes are in the database. \fixme{skriv til backend/servershit}

The project in the state it is currently in, seems a little bit unfinished. It could most definitely use some polishing and new graphics on the client side, the graphics currently used are purely android studio sprites. \fixme{more shit about the frontend/android shitthing}

