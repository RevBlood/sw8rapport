\section{Future Work}
\label{sec:future}

While the project as entirety is runnable and somewhat useful, it is also far from being finished. Several features are missing and some existing features need improvements. This section contains our thoughts on what should be be considered in the future, if the project was to be worked upon further.

\subsection{Backend}

\paragraph{Improved recipe data} 
We would have loved to have some data to further describe the recipe and through that improve the recommendation system. Some good examples here could be tags like organic or some more detailed descriptions like calories. Some of the most interesting information for our target group could be preparation time and calorie intake. It does however require some user interaction or some sort of admin checking all the recipes. Optimally we would have like to have a serious and highly active group of users and, as we will describe later, some sort of standardized way to input recipes. It does require a lot of maintenance and adaptation of the already existing recipes. Ideally we would want as detailed recipe information as possible, because it does give us a more precise and better recommendations.

\paragraph{Better ingredient matching}
The ingredient matching is very lackluster. It does not currently insert offers, so that they can be used in cooperation with ingredients to calculate the cost of cooking a given recipe. The matching has to be extremely precise to avoid having false-positive matching, which will make the calculation of cost wrong and the user-experience will decrease. 
The idea we have goes through a lot of phases, and sometimes iterates over the phases more than onces. Firstly we would check if the offer even slightly resembles anything in the database, and if it does not just throw it away. The second phase makes percentile comparison of the existing data, and through this we can decide which offers match. This algorithm is something a developer could use years to perfect, and something that probably never would be 100\% perfect.

\paragraph{Improved recommendation system}
This is one of the more wider suggestions and probably the suggestion which is most open to interpretation. The recommendation system as it is implemented is not really that great as we already touched on in the evaluation. We have already mentioned ways to improve the system we have in place with simple mathematical solutions like standard deviation.

This suggestion is highly based on how the recipe data would have been improved. The more detailed the ingredient, recipe and offer data is, the better the recommendation system is. Something we have not discussed is the ability to log the users earlier recipe choices and take these into consideration when recommending. Imagine a person who eats a lot of spaghetti dishes but rarely eat rice dishes, we would like to rate spaghetti dishes far higher in his recommendations.

\paragraph{Indexing of database}
The database is completely plain at this moment. We would like to have to made a simple indexing in the form of a B-tree indexing. This is by far the most used form of database indexing. With our rather limited time and database understanding, it seems highly unlikely that a custom-fitted database indexing would be beneficial. Having an indexed database would speed up the performance of querying significantly, which is something we have put a lot of focus on when designing the database.

\paragraph{Automation of system} 
As per this instance, we ran the setup (database updating and eTilbudsAvis data) once per each computer and worked with the data given. It was clear from the beginning we wanted to create a fully automated update system, that once a night would gather all the information and matched the data. We assumed that the clientèle would not use the application at this point, and since the data from eTilbudsAvis is stale and rarely change, it seems like a solid solution.

\paragraph{Support for pictures}
This is self-explanatory, the database does not a this point support pictures. We would have liked to implement a way for the users to upload pictures and store them in a cheap manner. This would require some good encoding and encryption of the pictures, as it could easily use excessive amounts of memory if the pictures are of a high quality.

\paragraph{Automatic store recognition and recommendation}
The idea is pretty simple, whenever you walk into a store from the database the application could pop up with suggestions and recommendations based on your preferences. This idea could be implemented with our current system pretty easily. Since we have all the locations of the stores we could query the database with a radius of 0.03 (30 meters), and get the suggestions. This means we would have to constantly check whether all the users are in the same locations as any of the stores in the database. However we would have to liked to have a partnership with retailers and stores, and have the application force a notification whenever you entered a store (preferably by wifi). This would make it possible to just query the database on the specific store and get a fast and precise recommendation.

\subsection{Frontend}

\paragraph{Displaying help for network communication}
One of the few things that can go wrong in the application is network communication. Unstable networks, unreachable server, or other factors between the smartphone and the server can mean a failure in communication. Currently the application is built around the assumption that a server is always available - as such there is no visual guides for the user, and with the exception of the \texttt{LoginActivity}, there is no indication of loading or errors that occur. Also, requests do not time out, providing no options to retry. Before releasing the application, these things are very important to have.

\paragraph{Paginate recipes and comments}
There are several reasons that recipes and comments should be paged. Not doing so means the application can run out of memory, trying to display too much data, or in a better scenario just be resource heavy. There is however also a problem at the usability level. It would be very difficult to retain an overview of recipes or comments when the size grows much larger than the height of the screen. 

\paragraph{Handle image scaling properly}
Profile pictures are currently always scaled identically if above a threshold. This should be updated to be done dynamically according to the dimensions of images. Not doing so can result in failing to import the image (out of memory).

\paragraph{Inputting new recipes through GUI}
The functionality to submit new recipes already exists. It is available on the backend, and also set up for use in the application. There is however no way for a user to reach the functionality. This is something that should be designed in the future as a new \texttt{Activity}. User submitted content is a noteworthy feature and was not meant to be left out.

\paragraph{Retouching the visual styles}
Given the focus of functionality and user-friendly designs, visual styles was less of a priority. In effect the application is easy to use, but very dull to look at. Before releasing the application, it should be considered to add color themes, graphic effects and proper icons/images. Currently, most elements use default Android colors.

\paragraph{Updating the Discover fragment}
It was intended that the \texttt{DiscoverFragment} should be a gateway to browse the database for recipes outside of the recommendation system. This was not done, and should be created. A possible implementation could be alphabetical lookup with the possibility to search free-text or by ingredients or tags.

\subsection{Other}

\paragraph{Researching price data}
One of the harder challenges in this project was finding usable data for ingredients and their availability. The current use of eTilbudsAvisen allows to see offers and their price, while the statistics from Statistics Denmark gives rough ideas on normal prices.
These data are however still a fraction of all buyable ingredients. As future work it would be a good idea to look into how better information on ingredients could be gathered.
One idea would be to crowdsource the information, encouraging users of the application to write down, through the application, what they have bought, what it cost, where it is from, and maybe tag the ingredient with relevant tags. This could help reinforce the database greatly.