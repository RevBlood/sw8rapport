\section{Existing Software}
\label{sec:exsoft}

As mentioned there are a couple of existing software solutions that solves parts of the problem base we have earlier established. There are a lot of different recipe applications or websites and to mention a few; Food Network in the Kitchen \cite{recipe_FN}, Woman's Day 100 Budget Meals \cite{recipe_woman}, Slow Copper Recipes \cite{recipe_SC} and Supercook \cite{recipe_supercook}. What all of these have in common is that they provide recipes based on certain premises, whether it is based on certain equipment such as a slow cooker or it is based on cheap meals. \\
Neither of these have a location based system -in the best case, they only provide recipes and a estimate of the costs based on online stores. All of the above mentioned are however American examples, and of course there are plenty of danish applications built on the same concept. In America it is possible to find prices of staple goods online as it is possible to order things from regular grocery retailers, but that is not possible in Denmark. It is sometimes possible to search by brand in many of the applications but none of them have automatic comparison of nearby grocery stores. In Denmark we have an existing service, the before mentioned eTilbudsavis \cite{etilbudsavis}. This service gathers information from catalogs the stores send out themselves, and tags all advertised products. It allows the user to create a wish list/shopping list and figure out if the items are on sale from different retailers. It furthermore uses maps to keep store locations, which means you have the possibility to see nearby stores and what they offer. This application does however not have recipe support, meaning you can not compare recipes and give estimates on the prices. As it is only based on offers, it is even difficult to give estimates on most full recipes, even if you wanted to.
