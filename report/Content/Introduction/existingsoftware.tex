\section{Existing Software}
\label{sec:exsoft}

There is a couple of existing software that solves parts of the problem base we have earlier established. There is a lot of different recipe applications or websites and to mention a few;  Food Network in the Kitchen \cite{recipe_FN}, Woman's Day 100 Budget Meals \cite{recipe_woman}, Slow Copper Recipes \cite{recipe_SC} and Supercook \cite{recipe_supercook}. What all of these have in common is that they provide recipes based on some premises, whether it is based on certain equipment such as a slow cooker or it is based on cheap meals. A lot of different big danish companies like \text{Arla}, \text{Amo} or \text{Dan Sukker} have custom fitted applications for their field of interests.\\
 Neither of these does however have a location based system - they only provide recipes and a estimate of the costs. All of these are however American examples, of course there are plenty of danish applications build on the same concept. It is possible to search by brand in many of the applications but none of them have automatic comparison of nearby grocery stores. In Denmark we have an existing similar service called eTilbudsavis \cite{etilbudsavis}. eTilbudsavis gathers information from the catalogs the stores send out themselves, and tags the ingredients respectively. It allows the user to create a wish list/shopping list and figures out if the items are on sale at the different retailers. It furthermore uses maps to locate stores, which means you have the possibility to see nearby stores and what they offer. This application does however not have recipe support, meaning you can not compare recipes and give estimates on the prices. As it is only based on offers, it is not even possible to give estimates on full recipes even if you wanted to.
