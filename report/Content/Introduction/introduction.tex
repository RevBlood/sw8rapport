\section{Introduction}
\label{sec:intro}

Having to plan dinner while on a tight schedule or on a tight budget can be a menace, especially if you crave something in particular. You would have to find a recipe online with the particular ingredient, you would have to figure out where to get those items, and where to get them the cheapest. Furthermore what if the rest of the dish is extremely expensive? then it does not really match up with the tight budget part of it.\\
It is not really hard to find an exciting recipe with ingredients that suit your taste as there are tons of different webservices and applications for that. The applications are usually simple and fast to use, however they are not always completely unbiased. Companies tend to launch big recipe applications for their specific product. An example could be the application \emph{`Karolines Køkken'}\cite{arla}, which recommends recipes with Arla products as ingredients. There are some great free applications with recipes, based on user uploaded recipes with a possibility to rate them such as \emph{`Opskrifter.dk'}\cite{opskrifterdk}. These applications provide easy lookup of recipes in different categories, but without price and ingredient searching.//
Specifically for searching on price but never by recipe, there are a couple of applications such as \emph{`eTilbudsavisen}\cite{etilbudsavisen}. These application gathers offers from different catalogs and gives you the cheapest ingredients in nearby stores. If a user wants to find the price of a certain recipe, he would have to search each ingredient individually and if they are not on sale, he has to know the price in a certain store. It is completely unreasonable to believe a person can remember every price for each product for each of the retailers.
