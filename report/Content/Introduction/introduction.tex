\section{Introduction}
\label{sec:intro}

Having to plan dinner while on a tight schedule or on a tight budget can be a menace, especially if you crave something in particular. In other situations, you just want a low-cost dinner to save money, especially if you are a student, it is the end of the month. But what recipes can you make?\\
It is not really hard to find an exciting recipe with ingredients that suit your taste as there are tons of different web services and applications for that. The applications are usually simple and fast to use, however they are not always completely unbiased. Companies tend to launch big recipe applications for their specific line of products. An example could be the application \emph{`Karolines Køkken'}\cite{arla}, which recommends recipes with Arla products as ingredients.\\
There are some great free applications, based on user uploaded recipes with a possibility to rate them such as \emph{`Opskrifter.dk'}\cite{opskrifterdk}. These applications provide easy lookup of recipes in different categories, but without price and ingredient searching.//
Applications for searching on groceries or ingredients exist, but they don not incorporate recipes. There are a couple of applications such as \emph{`eTilbudsavisen}\cite{etilbudsavisen}. These application gathers offers from different catalogs and gives you the cheapest ingredients in nearby stores. If a user wants to find the price of a certain recipe, he would have to search each ingredient individually. If the ingredients are not on sale, he has to know the price in a certain store beforehand. It is completely unreasonable to believe a person can remember every price for each product for each of the retailers.
