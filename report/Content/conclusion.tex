\chapter{Conclusion}
\label{chap:conc}

The final product has turned out somewhat functional. it is able to provide suggestions based on users preferences and allows for easily obtained information about nearby stores and retailers. The problem statement described in \ref{sec:probstate} has in the clearest meaning been solved as the application is able to provide suggestions based on the preferences set by the specific user. It is however not ideal that the only preference to set at this given moment is monetary. The application could be greatly improved by having support for more preferences with more diversity so the target group could be broadened. We have as mentioned made it possible for each user to determine what the ideal radius for shopping is for them - this is perfect for fitting all target groups when it comes to means of transportation, whether by car, walking or other.

It should be noted that while serving recommendations are completely possible, the final product suffers greatly from a lack of data. At this point there are no available recipes unless entered manually, ingredients only from offers and statistics, and this makes it hard to actually make good recommendations.
Furthermore, we wanted to tag recipes to provide a form of meta-data that would improve the recommendation system, but again, this kind of information proved difficult to obtain without entering it manually.

The server has been developed with performance in mind. We have intentionally created some suboptimal spacial entities with the purpose of making certain queries faster. These queries are generally the ones used by the application, meaning that these queries are likely to be run often. The server does however spend some time making the calculations depending on how many recipes are in the database. The database was made accessible through a REST API, which was found to work very well. It was easy to access from the application, and also great at abstracting the database away. From the server side, extensive error handling is provided, meaning that if something should go wrong, the backend can recover, and will provide proper error messages for the application to deal with. The application is however responsible for handling cases where something goes wrong, but was not implemented in time. This makes the clients communication naive, and prone to fail.

The application is unfinished although runnable. It could most definitely use some polishing and new graphics, as the graphics currently used are purely android studio sprites. There are both unfinished components, visuals and elements that needs refinement. It is however able to provide an intuitive way of getting recommendations through ones own preferences, along with browsing the contents of a recipe. The goal of making it user-friendly, feature rich and stable was mostly met. Features are limited given the limited project period, but extensive enough to provide a somewhat useful application. A lot of thought has gone into the usability of the application, and as result it feels easy to use. It is however not stable, as a faulty attempt to communicate with the server could crash the application in its current state.

\section{Evaluation}
\label{sec:eval}

The following paragraph will evaluate on the before mentioned conclusion. It gives more in depth evaluation than the conclusion, and digs deeper into the problematics several aspects of the program entails.

\paragraph{Recommender System}
\label{para:recommend}

The recommendation system, based on preferences is described in section \ref{sec:searchbypref}. The system has some obvious flaws that can be worked around, however designing a system like this is all about trade-offs. Our system could however be improved without introducing new trade-offs, as per right now the system gives a flat value based on its position in a sorted list (by the specific value). This creates scenarios where the actual value does not matter as much, imagine a recipe for which a retailer has an absurdly cheap offer compared to all the others. This recipe will be given a value of 100, while the next recipe which might be substantially more expensive, will be given a value of 99 or even 100 as well. While the system we have created does give a larger rating to the a cheaper offer, it does not give a proper representation of the relationship between values.

A way to solve this could be by implementing a standard deviation solution. It would be possible to give points based on how much a retailers offer differs from the actual average of the entire bunch of offers. This would make a lot more sense as the points given really reflect the value rather than the position of the value in a sorted list.

\subsection{Backend}
\label{subsec:evalbackend}
\paragraph{RESTful Web Service}
The web service performs very well and fulfills the purpose it was designed for. It offers a total of 46 HTTP requests to the client and completely hides away complex database functionality, as well as return useful computational results for the client like the results from the recommendation system. 

The web service is well structured and has been split into self-documenting interfaces responsible for very separated parts of the backend. Following this strict name-convention, it should be clear for the developer using the web service what a specific request should return. If something goes wrong, they are provided with a informative error-message, as can be seen in section \ref{chap:test} regarding test. 

In this project the web service does not fully receive the recognition it deserves. This is due to the uniform interface is provides, which is completely platform independent, meaning that this backend could serve iOS, windows phone or in this case android applications without alterations. This goal was part of the initial design decisions for the RESTful web service, because we wanted to develop a long-term realistic solution instead of hard-coding Android-specific responses based on an ad-hoc developed interface. This was not attractive for this project and complements the creation of the web service.

One of the only downsides to the RESTful web service is the lack of performance testing. We designed the backend with a purpose for fast query-handling and wanted to decrease the time a user spends waiting for results in the application. We never reached a goal with the data management that allowed us to proceed into this type of testing. The amount of data stored would simply not reflect the correct query time and the response-time through the web service would not be realistically used to say anything about the performance measures.

\paragraph{Data Management}
This project is highly dependent on the use of good and valid data. The data management has been prioritized very important which is why so many resources has been spent making the right decisions with the database. We wanted to avoid having to redesign the database because so much other functionality relies on it and is built on top of it. The RESTful web service HTTP requests are functioning purely as gateway to the database, and database changes cascades through the entire backend and requires extensive alterations, also to the web service. 

We coded an approximately 700 line long \texttt{DBController}, which handles all the desired database queries in this project. Had this project continued, we would spend time looking upon these queries and performance-optimizing them to deliver the fastest results possible to the user. Due to the limited resources in this project, we decided to lay aside the optimization until we had realistic data and application usage to work with. 

We only had to make two minor adjustments to the initial database design in this project. Because we spent so many resources designing it, we were able to keep the number of adjustments this low. The first adjustment we had to make was removing tags from recipes and ingredients themselves and making this as a weak-entity type, allowing easier and faster querying using tags. The second adjustment was an update to the attributes on ingredients. We added three additional attributes called organic, fat, and fresh. Since we started working with the offers from eTilbudsavis\cite{etilbudsavis}, we found that adding these attributes would make querying for healthy or fresh food much easier. Therefore we implemented this update.

\paragraph{Completeness}
The backend as a data provider is fully functional, and it is easily possible to access the data through the web service. The problem with the backend is  getting the valuable data from the external sources, parsing the data correctly and storing in the database for the user to look at and use. This is the biggest downside to the backend in its current state.
\chapter{Frontend}
\label{chap:frontend}

Intro!



\section{Future Work}
\label{sec:future}

\subsection{Backend}

\paragraph{Improved recipe data}
Så man kan judge øko, sundt, hurtigt at lave osv.

\paragraph{Better ingredient matching}
Så man bedre kan matche tilbud til ingredienser

\paragraph{Improved recommendation system}
Så det tager højde for flere ting

\paragraph{Recommendation categories}
Support for andet end aftensmad - kager, desserter, forretter osv.

\paragraph{Indexing of database}
Efter bogstaver m.m.

\paragraph{Automation of system}
Auto-opdateringer af alting

\paragraph{Support for pictures}
Opskrifer og profilbillede.

\subsection{Frontend}

\paragraph{Inputting new recipes in GUI}

\paragraph{Retouching the visual styles}
Appen er tør og kedelig

\paragraph{Updating the Discover fragment}
Der skal noget søgefunktion og noget til.

\subsection{Other}

\paragraph{Researching price data}
Find flere kilder, eller lav crowdsourcing på brugere.

